\documentclass{article}
\usepackage[portuguese]{babel} % this translates commands like \today into portuguese language!
\usepackage[utf8]{inputenc}
\usepackage[T1]{fontenc}

\begin{document}
\title{Proposta de Trabalho para Programação Concorrente Orientada por Objectos}
\author{André Prata (andreprata@ua.pt)}
\date{\today}

\maketitle

\section{Proposta}
Como projecto final da disciplina de Programação Concorrente Orientada por
Objectos proponho a realização de um jogo 2D baseado no conceito do Pac-Man.

O espaço a duas dimensões será representado por um tabuleiro reticulado, com
suporte ao package pt.ua.gboard desenvolvido pelo professor Miguel Oliveira e
Silva. Intrínseca a este reticulado está uma representação do mundo e o seu
estado, o \textbf{Mapa}, potencialmente partilhada pelas restantes entidades.
De entre estas, destacam-se:

\begin{itemize}
\item O \textbf{Jogador}. É a entidade independente na dinâmica do Jogo. Pode
ser uma pessoa, mas também um sistema de informação ou outro módulo de software.

\item O \textbf{Pac-Man}. O Pac-Man é a entidade que o jogador controla. Por não
se tratar necessariamente de uma pessoa ou teclado, o Pac-Man é um actor que
exporta uma interface de atendimento a pedidos relativos ao movimento e outras
acções que possa tomar.

\item O \textbf{Fantasma}, inimigo do Pac-Man. Trata-se de uma entidade pouco
dotada de inteligência que se desloca pelo mapa em busca do Pac-Man. Caso se
encontrem no mesmo espaço, a fatalidade é inevitável e o jogo acaba. Pode
existir mais do que um Fantasma por mapa num mesmo instante. A sua existência é
permanente no decorrer do jogo.

\item Os \textbf{Feijões} estão espalhados pelo mapa. Um nível termina quando o
Pac-Man consome todos os Feijões.

\item Os \textbf{Bónus} aparecem aleatoriamente no mapa, e por tempo limitado.
Se os apanhar, o Pac-Man ganha pontuanção extra ou vantagens relativamente aos
inimigos (pode movimentar-se mais depressa, os inimigos mais lentamente, ou
pode simplesmente tornar-se imune). Deverá ser o Bónus a provocar estas
mutações no mundo, e das quais as restantes entidades dependem.
\end{itemize}

\section{Considerações Adicionais}
A dinâmica de jogo pode complicar-se sistematicamente. Por exemplo, se o
Pac-Man passar muito tempo sem consumir Feijões duplica-se noutra zona do mapa.
O Jogador passa a controlar ambas as instâncias concorrentemente e da mesma
forma. Assim torna-se mais simples capturar mais Feijões mas também se aumenta
a probabilidade de chocar contra um Fantasma. Esta colisão pode ditar o fim do
Jogo ou apenas de uma das instâncias do Pac-Man. Poderão ser considerado também
um número finito de vidas para o Pac-Man, incrementado com boas prestações do
Jogador.

Os Mapas deverão ser criados e armazenados num ficheiro de texto. Está aberta a
possibilidade para elementos mutáveis no mapa, como portas entre áreas
diferentes.

Será necessário pensar numa forma eficaz de sincronizar os eventos de forma a
que o mapa se actualize a uma taxa constante e à qual todas as entidades
conseguem ser atendidas caso necessitem de realizar tarefas.

\ldots
\end{document}
